\documentclass[12pt,a4paper,]{report}

\usepackage[T1,T8K,T8M]{fontenc}
\usepackage[utf8]{inputenc}
\usepackage[english,georgian]{babel}

\usepackage{graphicx}
\usepackage{hyperref}
\usepackage{mathtools}
\usepackage{gensymb}

%\renewcommand{\rmdefault}{cmtt}

\hypersetup{
	pdftitle={dummy},
	pdfauthor={Levan Kankadze},
	bookmarksnumbered=true,     
	bookmarksopen=true,         
	bookmarksopenlevel=1,       
	colorlinks=true,
	linkcolor=blue,            
	pdfstartview=Fit,           
	pdfpagemode=UseOutlines,    
	%pdfpagelayout=TwoPageRight
}

\textwidth=16.5cm
\textheight=25cm
\voffset=-2.5cm
\hoffset=-2cm

 %ტექსტის გვერდების გასწორება
\tolerance=10000
\hyphenpenalty=100

\begin{document}

	\section{რა არის რადაციული თერაპია?}
რადიაციული თერაპია არის სიმსივნის მკურნალობის მეთოდი, თუმცა დღესდღეობით რადიაციული თერაპიით შესაძლებელია სხვა დაავადებების განკურნებაც (გული, ...). რადიაციული თერაპია იყენებს ინტენსიური ნაკადების (დამუხტული ნაწილაკების ანდა ელექტრომაგნიტური გამოსხივების) ენერგიას. ხშირად რადიაციული თერაპია იყენებს რენტგენის სხივებს, თუმცა პროტონების ან სხვა დამუხტული ნაწილაკების გამოყენებაც შეიძლება.
	
	\section{რადიაციული ერთეულები და დოზები}
როდესაც გამოსხივება (დამუხტული ნაწილაკების ანდა ფოტონების) გადის ნივთიერებაში ურთიერთქმედებს ნივთიერების ატომებთან. რადიაციული თერაპიისას ასეთი ნივთიერებად პაციენტის სხეული განიხილება. ამ ურთიერთქმედების შედეგად ნაწილაკები ტოვებენ ენერგიას გარემოში. დატოვებული ენერგია ნივთიერებაში რიცხვითად ხასიათდება როგორც მიღებული დოზა. 
	არსებობს შემდეგი ტიპი დოზების:
	\begin{description}
      \item[$\bullet$] შთანთქმული დოზა  (absorbed dose)
      \item[$\bullet$] ექვივალენტური დოზა
      \item[$\bullet$] ეფექტური დოზა (effective dose)
    \end{description}
    
    Absorbed dose is defined as the energy deposited by ionizing radiation per unit
mass of material and is expressed in J
kg . This unit represents Gy-gray or 1 J
kg .
Equivalent dose is dened as the absorbed dose multiplied with the radiation
weight factor.
    
\textit{შთანთქმული დოზა} განისაზღვრება როგორც მაიონიზერებელი გამოსხივების მიერ დატოვებული ენერგია ნივთიერების ერთეულ მასაზე და გამოისახება როგორც $\frac{\text{ჯ}}{\text{კგ}}$
    \begin{table}[htp]
        \centering
        \begin{tabular}{l | l}
             დასხივების ტიპი & დასხივების "წონა" \\
             \hline
             \hline
             რენტგენი & 1 \\
             $\gamma$-zrake(?) & 1 \\
             ელექტრონები და პოზიტრონები & 1 \\
             ნეიტრონები & Energy dependence \\
             2 მევ-ის პროტონები & 2 \\
             $\alpha$ ნაწილაკები და მძიმე იონები & 20 \\
        \end{tabular}
        \caption{Caption}
        \label{tab:my_label}
    \end{table}

\section{რადიაციული დაზიანება}
სხვადასხვა ენერგიის და სახის გამოსხივება სხვადასხვანაირად მოქმედებს სხეულში. დაბალი ენერგიის ნაწილაკებს გააჩნიათ უფრო დაბალი განჭოლვის უნარი. ამავდროულად სხვადასხვა სახის ურთიერთქმედება იწვევს სხვადასხვა სახის დაზიანებას ცოცხალი ორგანიზმის უჯრედებში. გამოსხივება პირდაპირ მოქმედებს დნმ-ზე. 
დნმ-ი შედგება ორი დაკავშირებული პოლინუკლუედური ჯაჭვისგან და წარმოქმნის სპირალს. რადიაციის შედეგად ზიანდება ეს ჯაჭვები და ამის შედეგად შეიძლება უჯრედი სრულად აღდგეს, ან არასწორად აღდგეს ანდა მოკვდეს. ჯანმრთელი უჯრედების დასხივებისას ყველაზე სასურველია პირველი შემთხვევა, თუმცა უფრო ხშირად მეორე ან მესამე შემთხვევა ვითარდება. მეორე შემთხვევა ყველაზე საშიშია რადგანაც, არასწორად აღდგენილი, მუტირებული უჯრედმა შესაძლოა სიმსივნე გამოიწვიოს. 

მძიმე იონების და პროტონებით დასხივებისას ზიანდება ორივე ჯაჭვი და იწვევს უჯრედის სრულ სიკვდილს, ამიტომაც უჯრედის მუტაცია აღარ ხდება, ფოტონებით დასხივებისას ზიანდება მხოლოდ ერთი ჯაჭვი რაც ტოვებს უჯრედის მუტაციის რისკს. ამავდროულად სიმსივნურ უჯრედებს არ გააჩნიათ აღდგენის უნარი და დნმ-ის დაზიანებისას ისინი კვდებიან, მაგრამ გარკვეულ შემთხვევებში სიმსივნე მედეგია ფოტონური დასხივების მიმართ. ამ მიზეზთა გამო პროტონებსა და ნახშირბადის ბირთვებს აქვთ მეტი ალბათობა სიმსივნური უჯრედების განადგურებისა.

	\begin{figure}[htp]
	    \centering
        \includegraphics[width = 6cm]{images/Radiotherapy.jpg}
        \caption{caption.}
        \label{fig:1}
    \end{figure}

    \subsection{RBE (relative biological effectiveness)
 ფბე (ფარდობითი ბიოლოგიური ეფექტურობა) } 

	\begin{figure}[htp]
	    \centering
        \includegraphics[width = 8cm]{images/Picture1.png}
        \caption{caption.}
        \label{fig:1}
    \end{figure}

\section{რადიაციული თერაპიის დაგეგმვა (Radiation Treatment Planing)}
\subsection{ფანტომები (Phantoms)}

\subsection{მინიმალური სამიზნე დოზა (Minimum Target Dose)}
მინიმალური სამიზნე დოზა არის სამიზნე ფართობის მიერ მინიმალური შთანთქმული დოზა.

\subsection{საშუალო სამიზნე დოზა (Media Target Dose)}
საშუალო სამიზნე დოზა არის სამიზნის მიერ შთანთქმული მაქსიმალური და მინიმალური დოზის საშუალო მნიშვნელობისა.  

\subsection{ცხელი წერტილები (Hot Spots)}
ცხელი წერტილი არის ფართობი რომელიც არის მიზნის გარეთ და იღებს უფრო მეტ დოზას ვიდრე მიზნისთვისაა განსაზღვრული. როგორც მაქსიმალური სამიზნე დოზა, ცხელი წერტილი იძენს სამედიცინო აზრს თუ ის ფარავს სულ მცირე 2~$\text{სმ}^2$ ფართობს.

\medskip

%\printbibliography

\end{document}